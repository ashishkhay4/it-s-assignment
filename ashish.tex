\documentclass[12pt]{article}

\title{National Institute of Technology,Raipur\\ASSINGMENT-1 of Biomedical Engineering}
\author{Submitted by:Ashish\\Roll.no:21111013\\Details:1st Semester,Biomedical Engineering\\Supervision of:Saurabh Gupta Sir}

\begin{document}
\maketitle
\clearpage
1st. Topic name:AIR CLEANER


\section{What is Air Cleaner}
An air purifier or air cleaner is a device which removes contaminants from the air in a room to improve indoor air quality. These devices are commonly marketed as being beneficial to allergy sufferers and asthmatics, and at reducing or eliminating second-hand tobacco smoke.

The commercially graded air purifiers are manufactured as either small stand-alone units or larger units that can be affixed to an air handler unit (AHU) or to an HVAC unit found in the medical, industrial, and commercial industries. Air purifiers may also be used in industry to remove impurities from air before processing. Pressure swing adsorbers or other adsorption techniques are typically used for this.

\section*{History}
In 1830, a patent was awarded to Charles Anthony Deane for a device comprising a copper helmet with an attached flexible collar and garment. A long leather hose attached to the rear of the helmet was to be used to supply air, the original concept being that it would be pumped using a double bellows. A short pipe allowed breathed air to escape. The garment was to be constructed from leather or airtight cloth, secured by straps

\section{Use and Benefits}
Dust, pollen, pet dander, mold spores, and dust mite feces can act as allergens, triggering allergies in sensitive people. Smoke particles and volatile organic compounds (VOCs) can pose a risk to health. Exposure to various components such as VOCs increases the likelihood of experiencing symptoms of sick building syndrome.

\section{Purifying Techniques}
There are two types of air purifying technologies, active and passive. Active air purifiers release negatively charged ions into the air, causing pollutants to stick to surfaces, while passive air purification units use air filters to remove pollutants. Passive purifiers are more efficient since all the dust and particulate matter is permanently removed from the air and collected in the filters.

Several different processes of varying effectiveness can be used to purify air. As of 2005, the most common methods were high-efficiency particulate air (HEPA) filters and ultraviolet germicidal irradiation (UVGI).

\section{Filtration}
Air filter purification traps airborne particles by size exclusion. Air is forced through a filter and particles are physically captured by the filter.

\section{Other Methods}
UVGI can be used to sterilize air that passes UV lamps via forced air. Air purification UVGI systems can be freestanding units with shielded UV lamps that use a fan to force air past the UV light. Other systems are installed in forced air systems so that the circulation for the premises moves micro-organisms past the lamps. Key to this form of sterilization is the placement of the UV lamps and a good filtration system to remove the dead micro-organisms. For example, forced air systems by design impede line-of-sight, thus creating areas of the environment that will be shaded from the UV light

Activated carbon is a porous material that can adsorb volatile chemicals on a molecular basis, but does not remove larger particles. The adsorption process when using activated carbon must reach equilibrium thus it may be difficult to completely remove contaminants.

Polarized-media electronic air cleaners use active electronically enhanced media to combine elements of both electronic air cleaners and passive mechanical filters. Most polarized-media electronic air cleaners use safe 24-volt DC voltage to establish the polarizing electric field.

Plasma air purifiers are a form of ionizing air purifier.


\maketitle
\clearpage
2nd.Topic Name:X-Ray Tube

\section*{1. What is X-Ray Tube}
An X-ray tube is a vacuum tube that converts electrical input power into X-rays.[1] The availability of this controllable source of X-rays created the field of radiography, the imaging of partly opaque objects with penetrating radiation. In contrast to other sources of ionizing radiation, X-rays are only produced as long as the X-ray tube is energized. X-ray tubes are also used in CT scanners, airport luggage scanners, X-ray crystallography, material and structure analysis, and for industrial inspection.

Increasing demand for high-performance Computed tomography (CT) scanning and angiography systems has driven development of very high performance medical X-ray tubes.

\section*{2. History }
Until the late 1980s, X-ray generators were merely high-voltage, AC to DC variable power supplies. In the late 1980s a different method of control was emerging, called high speed switching. This followed the electronics technology of switching power supplies (aka switch mode power supply), and allowed for more accurate control of the X-ray unit, higher quality results, and reduced X-ray exposures.

\section*{3. PHYSICS}
As with any vacuum tube, there is a cathode, which emits electrons into the vacuum and an anode to collect the electrons, thus establishing a flow of electrical current, known as the beam, through the tube. A high voltage power source, for example 30 to 150 kilovolts (kV), called the tube voltage, is connected across cathode and anode to accelerate the electrons. The X-ray spectrum depends on the anode material and the accelerating voltage.

Electrons from the cathode collide with the anode material, usually tungsten, molybdenum or copper, and accelerate other electrons, ions and nuclei within the anode material. 


\section*{4. TYPE\\Crookes tube:}
They had an aluminum cathode plate at one end of the tube, and a platinum anode target at the other end. The anode surface was angled so that the X-rays would radiate through the side of the tube. The cathode was concave so that the electrons were focused on a small (~1 mm) spot on the anode, approximating a point source of X-rays, which resulted in sharper images. The tube had a third electrode, an anticathode connected to the anode. It improved the X-ray output, but the method by which it achieved this is not understood. A more common arrangement used a copper plate anticathode (similar in construction to the cathode) in line with the anode such that the anode was between the cathode and the anticathode.

\section*{Coolidge tube}
In the Coolidge tube, the electrons are produced by thermionic effect from a tungsten filament heated by an electric current. The filament is the cathode of the tube. The high voltage potential is between the cathode and the anode, the electrons are thus accelerated, and then hit the anode.

\section*{5. Rotating anode tube}
A considerable amount of heat is generated in the focal spot (the area where the beam of electrons coming from the cathode strike to) of a stationary anode. Rather, a rotating anode lets the electron beam sweep a larger area of the anode, thus redeeming the advantage of a higher intensity of emitted radiation, along with reduced damage to anode compared to its stationary state.The molybdenum conducts heat from the target.

 The graphite provides thermal storage for the anode, and minimizes the rotating mass of the anode.
 
 
 
 \maketitle
 \clearpage
 3rd. Topic Name:  Water Purification System
 
 \section*{1.  What is Water Purification System }
 Water purification means the process of removing undesirable chemicals, biological contaminants, suspended solids, and gases from water. The goal is to produce water that is fit for specific purposes. Most water is purified and disinfected for human consumption (drinking water), but water purification may also be carried out for a variety of other purposes, including medical, pharmacological, chemical, and industrial applications. The history of water purification includes a wide variety of methods. The methods used include physical processes such as filtration, sedimentation, and distillation; biological processes such as slow sand filters or biologically active carbon; chemical processes such as flocculation and chlorination; and the use of electromagnetic radiation such as ultraviolet light.
 
 \section*{2.  Source of Water}
 Groundwater: The water emerging from some deep ground water may have fallen as rain many tens, hundreds, or thousands of years ago. Soil and rock layers naturally filter the ground water to a high degree of clarity and often, it does not require additional treatment besides adding chlorine or chloramines as secondary disinfectants. Such water may emerge as springs, artesian springs, or may be extracted from boreholes or wells. Deep ground water is generally of very high bacteriological quality (i.e., pathogenic bacteria or the pathogenic protozoa are typically absent), but the water may be rich in dissolved solids, especially carbonates and sulfates of calcium and magnesium.
 
\section*{3.  Goals}
The goals of the treatment are to remove unwanted constituents in the water and to make it safe to drink or fit for a specific purpose in industry or medical applications. Widely varied techniques are available to remove contaminants like fine solids, micro-organisms and some dissolved inorganic and organic materials, or environmental persistent pharmaceutical pollutants. The choice of method will depend on the quality of the water being treated, the cost of the treatment process and the quality standards expected of the processed water.

The processes below are the ones commonly used in water purification plants. Some or most may not be used depending on the scale of the plant and quality of the raw (source) water.

\section*{4.  Filtration}
The most common type of filter is a rapid sand filter. Water moves vertically through sand which often has a layer of activated carbon or anthracite coal above the sand. The top layer removes organic compounds, which contribute to taste and odour. The space between sand particles is larger than the smallest suspended particles, so simple filtration is not enough. Most particles pass through surface layers but are trapped in pore spaces or adhere to sand particles. Effective filtration extends into the depth of the filter. This property of the filter is key to its operation: if the top layer of sand were to block all the particles, the filter would quickly clog.

\section*{5.  Disinfection}
Disinfection is accomplished both by filtering out harmful micro-organisms and by adding disinfectant chemicals. Water is disinfected to kill any pathogens which pass through the filters and to provide a residual dose of disinfectant to kill or inactivate potentially harmful micro-organisms in the storage and distribution systems. 

The most common disinfection method involves some form of chlorine or its compounds such as chloramine or chlorine dioxide. Chlorine is a strong oxidant that rapidly kills many harmful micro-organisms. Because chlorine is a toxic gas, there is a danger of a release associated with its use.



\maketitle
\clearpage
4th. Topic Name:  VACCUME PUMP 

\section*{1.  what is Vaccume Pump}
A vacuum pump is a device that draws gas molecules from a sealed volume in order to leave behind a partial vacuum. The job of a vacuum pump is to generate a relative vacuum within a capacity. The first vacuum pump was invented in 1650 by Otto von Guericke, and was preceded by the suction pump, which dates to antiquity.

\section*{2.  History}
The predecessor to the vacuum pump was the suction pump. Dual-action suction pumps were found in the city of Pompeii. Arabic engineer Al-Jazari later described dual-action suction pumps as part of water-raising machines in the 13th century. He also said that a suction pump was used in siphons to discharge Greek fire. The suction pump later appeared in medieval Europe from the 15th century.

By the 17th century, water pump designs had improved to the point that they produced measurable vacuums, but this was not immediately understood. What was known was that suction pumps could not pull water beyond a certain height: 18 Florentine yards according to a measurement taken around 1635, or about 34 feet (10 m). This limit was a concern in irrigation projects, mine drainage, and decorative water fountains planned by the Duke of Tuscany, so the duke commissioned Galileo Galilei to investigate the problem.

\section*{3.  Types\\ Positive Displacement Pump  }
Positive displacement pumps use a mechanism to repeatedly expand a cavity, allow gases to flow in from the chamber, seal off the cavity, and exhaust it to the atmosphere. Momentum transfer pumps, also called molecular pumps, use high speed jets of dense fluid or high speed rotating blades to knock gas molecules out of the chamber. Entrapment pumps capture gases in a solid or adsorbed state. This includes cryopumps, getters, and ion pumps.
 
 \section*{Momentim Transfer Pump }
 In a momentum transfer pump, gas molecules are accelerated from the vacuum side to the exhaust side (which is usually maintained at a reduced pressure by a positive displacement pump). Momentum transfer pumping is only possible below pressures of about 0.1 kPa. Matter flows differently at different pressures based on the laws of fluid dynamics. At atmospheric pressure and mild vacuums, molecules interact with each other and push on their neighboring molecules in what is known as viscous flow. When the distance between the molecules increases, the molecules interact with the walls of the chamber more often than with the other molecules, and molecular pumping becomes more effective than positive displacement pumping. This regime is generally called high vacuum.
 
\section*{4.  Techniques}
Vacuum pumps are combined with chambers and operational procedures into a wide variety of vacuum systems. Sometimes more than one pump will be used (in series or in parallel) in a single application. A partial vacuum, or rough vacuum, can be created using a positive displacement pump that transports a gas load from an inlet port to an outlet (exhaust) port. Because of their mechanical limitations, such pumps can only achieve a low vacuum. To achieve a higher vacuum, other techniques must then be used, typically in series (usually following an initial fast pump down with a positive displacement pump). Some examples might be use of an oil sealed rotary vane pump (the most common positive displacement pump) backing a diffusion pump, or a dry scroll pump backing a turbomolecular pump. There are other combinations depending on the level of vacuum being sought.



\maketitle
\clearpage
5th. Topic Name: VENTILATOR SYSTEM

\section*{1.   What is Ventilator System}
A ventilator is a machine that provides mechanical ventilation by moving breathable air into and out of the lungs, to deliver breaths to a patient who is physically unable to breathe, or breathing insufficiently. Ventilators are computerized microprocessor-controlled machines, but patients can also be ventilated with a simple, hand-operated bag valve mask. Ventilators are chiefly used in intensive-care medicine, home care, and emergency medicine (as standalone units) and in anesthesiology (as a component of an anesthesia machine).

Ventilators are sometimes called "respirators".

\section*{2.  Function }
In its simplest form, a modern positive pressure ventilator consists of a compressible air reservoir or turbine, air and oxygen supplies, a set of valves and tubes, and a disposable or reusable "patient circuit". The air reservoir is pneumatically compressed several times a minute to deliver room-air, or in most cases, an air/oxygen mixture to the patient. If a turbine is used, the turbine pushes air through the ventilator, with a flow valve adjusting pressure to meet patient-specific parameters. When over pressure is released, the patient will exhale passively due to the lungs' elasticity, the exhaled air being released usually through a one-way valve within the patient circuit called the patient manifold.

\section{3.   History  }
The history of mechanical ventilation begins with various versions of what was eventually called the iron lung, a form of noninvasive negative-pressure ventilator widely used during the polio epidemics of the twentieth century after the introduction of the "Drinker respirator" in 1928, improvements introduced by John Haven Emerson in 1931, and the Both respirator in 1937. Other forms of noninvasive ventilators, also used widely for polio patients.


\section*{4.  Microprocessor Ventilators  }
Microprocessor control led to the third generation of intensive care unit (ICU) ventilators, starting with the Dräger EV-A in 1982 in Germany which allowed monitoring the patient's breathing curve on an LCD monitor. One year later followed Puritan Bennett 7200 and Bear 1000, SERVO 300 and Hamilton Veolar over the next decade. Microprocessors enable customized gas delivery and monitoring, and mechanisms for gas delivery that are much more responsive to patient needs than previous generations of mechanical.

\section*{5.  COVID-19 Pandemic  }
he COVID-19 pandemic has led to shortages of essential goods and services - from hand sanitizers to masks to beds to ventilators. Countries around the world have experienced shortages of ventilators. Furthermore, fifty-four governments, including many in Europe and Asia, imposed restrictions on medical supply exports in response to the coronavirus pandemic.

The capacities to produce and distribute invasive and non-invasive ventilators vary by country. In the initial phase of the pandemic, China ramped up its production of ventilators, secured large amounts of donations from private firms, and dramatically increased imports of medical devices worldwide. As a result, the country accumulated a reservoir of ventilators throughout the pandemic in Wuhan. Western Europe and the United States, which outrank China in their production capacities, suffered a shortage of supplies due to the sudden and scattered outbreaks throughout the North American and European continents. Finally, Central Asia, Africa, and Latin America, which depend almost entirely on importing ventilators, suffered severe shortages of supplies.

Healthcare policy-makers have met serious challenges to estimate the number of ventilators needed and used during the pandemic. When data is often not available for ventilators specifically, estimates are sometimes made based on the number of intensive care unit beds available, which often contain ventilators.

 
 
\end{document}









